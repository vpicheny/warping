%%%%%%%%%%%%%%%%%%%%%%% file template.tex %%%%%%%%%%%%%%%%%%%%%%%%%
%
% This is a general template file for the LaTeX package SVJour3
% for Springer journals.          Springer Heidelberg 2010/09/16
%
% Copy it to a new file with a new name and use it as the basis
% for your article. Delete % signs as needed.
%
% This template includes a few options for different layouts and
% content for various journals. Please consult a previous issue of
% your journal as needed.
%
%%%%%%%%%%%%%%%%%%%%%%%%%%%%%%%%%%%%%%%%%%%%%%%%%%%%%%%%%%%%%%%%%%%
%
% First comes an example EPS file -- just ignore it and
% proceed on the \documentclass line
% your LaTeX will extract the file if required
% \begin{filecontents*}{example.eps}
% %!PS-Adobe-3.0 EPSF-3.0
% %%BoundingBox: 19 19 221 221
% %%CreationDate: Mon Sep 29 1997
% %%Creator: programmed by hand (JK)
% %%EndComments
% gsave
% newpath
%   20 20 moveto
%   20 220 lineto
%   220 220 lineto
%   220 20 lineto
% closepath
% 2 setlinewidth
% gsave
%   .4 setgray fill
% grestore
% stroke
% grestore
% \end{filecontents*}
%
\RequirePackage{fix-cm}
%
%\documentclass{svjour3}                     % onecolumn (standard format)
%\documentclass[smallcondensed]{svjour3}     % onecolumn (ditto)
\documentclass[smallextended]{svjour3}       % onecolumn (second format)
%\documentclass[twocolumn]{svjour3}          % twocolumn
%
\smartqed  % flush right qed marks, e.g. at end of proof
%
\usepackage{booktabs}
\usepackage{graphicx}
\usepackage{times}
\usepackage{latexsym,mathrsfs}
\usepackage{amssymb,amsfonts,amsmath}
\usepackage[numbers]{natbib}
% \usepackage{algorithm}
% \usepackage{algpseudocode}
% \usepackage{algorithmicx}
\usepackage{color}
\usepackage{graphics}
\usepackage{graphicx}
\usepackage{bbm}
\usepackage{url}
\def\ds{\displaystyle}
\def\R{\mathbb{R}}
\newcommand{\x}{\mathbf{x}}
\newcommand{\X}{\mathbf{X}}
\newcommand{\y}{\mathbf{y}}
\newcommand{\f}{\mathbf{f}}
\newcommand{\Y}{\mathbf{Y}}
\newcommand{\F}{\mathbf{F}}
\newcommand{\z}{\mathbf{z}}
\newcommand{\s}{\mathbf{x}}
\newcommand{\Sset}{\mathbb{X}}
\newcommand{\Rset}{\mathbb{R}}
\newcommand{\Xset}{\mathbb{X}}
\newcommand{\Prob}{\mathbb{P}}

% packages and dependencies for colored comments in text (collab_tex is the package, the other ones are dependencies)
\usepackage[dvipsnames,svgnames]{xcolor}
\usepackage[normalem]{ulem}
\usepackage{collab_tex}

\begin{document}

\title{Warping%\thanks{Grants or other notes
%about the article that should go on the front page should be
%placed here. General acknowledgments should be placed at the end of the article.}
}

%\titlerunning{Short form of title}        % if too long for running head

\author{Victor Picheny         \and
        Coralie Picard        \and
        Gael Thebaud
}

%\authorrunning{Short form of author list} % if too long for running head

\institute{V. Picheny \at
              MIAT, Universit\'e de Toulouse, INRA, Castanet-Tolosan, France \\
              Tel.:  +33561285551\\
              \email{victor.picheny@inra.fr}           %  \\
%             \emph{Present address:} of F. Author  %  if needed
\and
           C. Picard \at
           BGPI, Montpellier SupAgro, INRA, Univ. Montpellier, Cirad, TA A-54/K, 34398, Montpellier 
           \and
           G. Thebaud \at to do
}

\date{Received: date / Accepted: date}
% The correct dates will be entered by the editor

\maketitle

\begin{abstract}
On peut \victor{faire un commentaire} \coralie{chacun avec sa couleur}, on peut aussi \victordelete{enlever des trucs} ou bien \coralieadd{ajouter d'autres trucs}, \gaeladd{et Gael aussi}.
\keywords{to do}
\end{abstract}

\section{Introduction}
\victor{toi ou moi}
Mathematical modelling for design, in particular in epidemiology

\victor{pour toi... mais peut-être plus facile à faire une fois que le reste aura avancé}

The sharka model and objectives

\victor{le reste de l'intro pour moi}

Generalities on optimization

Bayesian optimization

Problem at hand: dealing with local invariances

Outline

\section{Model description and problem set-up}
\victor{Section à remplir par toi ! Suggestion de plan détaillé.}

What does it model

How the model works

What problem do we want to solve

Inputs description

Invariances descriptions

Table of inputs with range of variation

Table of invariance relations

% Table generated by Excel2LaTeX from sheet 'Feuil3 (3)'
\begin{table}[htbp]
	\centering
	\caption{Add caption}
	\begin{tabular}{|c|p{33.785em}|c|c|}
		\cmidrule{3-4}    \multicolumn{1}{c}{} & \multicolumn{1}{c|}{} & \textbf{Min} & \textbf{Max} \\
		\midrule
		\multicolumn{4}{|c|}{\textbf{Epidemiological parameters}} \\
		\midrule
		$q_{K}$    & Quantile of the connectivity of the patch of first introduction & 0     & 1 \\
		\midrule
		$\phi$ & Probability of introduction at plantation (before management) & 0,02  & 0,02 \\
		\cmidrule{2-4}          & Probability of introduction at plantation (during management) & 0,0046 & 0,0107 \\
		\midrule
		$p_{MI}$ & Relative probability of massive introduction (before management) & 0,4   & 0,4 \\
		\cmidrule{2-4}          & Relative probability of massive introduction  (during management) & 0     & 0,1 \\
		\midrule
		$W_{exp}$  & Expected value of the dispersal weighting variable & 0,469 & 0,504 \\
		\midrule
		$\beta$     & Transmission coefficient & 1,25  & 1,39 \\
		\midrule
		$\theta_{exp}$  & Variance of the latent period duration (years) & 1,71  & 2,14 \\
		\midrule
		\multicolumn{4}{|c|}{\textbf{Management parameters}} \\
		\midrule
		$\rho$    & Probability of detection of a symptomatic tree & 0     & 0,66 \\
		\midrule
		$\gamma_{O}$    & Duration of observation zones (years) & 0     & 10 \\
		\midrule
		$\zeta_{s}$   & Radius-distance of security zones (m) & 0     & 5800 \\
		\midrule
		$\zeta_{f}$  & Radius-distance of focal zones (m) & 0     & 1 \\
		\midrule
		$\zeta_{eO}$ & Radius-distance of observation epicenter (m) & 0     & 1 \\
		\midrule
		1/$\eta_{0}$  & Maximal period between 2 observations (year) & 1     & 15 \\
		\midrule
		$\eta_{s}$    & Observation frequency in security zones (year-1) & 0     & 8 \\
		\midrule
		$\eta_{f}$    & Observation frequency in focal zones (year-1) & 0     & 8 \\
		\midrule
		$\eta_{f*}$   & Modified observation frequency in focal zones (year-1) & 0     & 8 \\
		\midrule
		$\chi_{o}$    & Contamination threshold in the observation epicenter, above which the observation frequency in focal zone is modified & 0     & 1 \\
		\bottomrule
	\end{tabular}%
	\label{tab:addlabel}%
\end{table}%



\section{Methods — Bayesian optimization}

\subsection{Overview}

\subsection{Bayesian optimization of stochastic simulators}

\subsection{Bayesian optimization with invariances}

\subsubsection{Definitions}

\subsubsection{Simple warping}

\subsubsection{Warping based on linear relations}

\subsubsection{Combining warpings}

\section{Experiments on toy problems}

\subsection{Problem descriptions}

\subsection{Comparison metrics}

\subsection{Results}

\section{A warping-based Bayesian optimization of the Sharka model}

\subsection{Numerical setup}

\subsubsection{Experiments description}

\victor{Premier jet par toi ?}

\subsubsection{Comparison with standard BO}

Description of comparison metrics

\victor{Idem juste pour les méthodes de comparaison, je me charge du paragraphe pour dire à quoi on se compare et je m'occupe de la partie krigeage et warping.}

\subsection{Results and insights into the Sharka model}

\section{Conclusion}

What we did (the problem we solved)

What we proposed: warping to tackle invariances. Proof of concept

Possible extensions

\section*{References}
\end{document}